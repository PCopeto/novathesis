%!TEX root = ../template.tex
%%%%%%%%%%%%%%%%%%%%%%%%%%%%%%%%%%%%%%%%%%%%%%%%%%%%%%%%%%%%%%%%%%%%
%% abstract-pt.tex
%% NOVA thesis document file
%%
%% Abstract in Portuguese
%%%%%%%%%%%%%%%%%%%%%%%%%%%%%%%%%%%%%%%%%%%%%%%%%%%%%%%%%%%%%%%%%%%%

\typeout{NT FILE abstract-pt.tex}%

%Esta dissertação apresenta o desenvolvimento, simulação e aplicação experimental de métodos de reconstrução para a deteção de fragmentos leves utilizando uma Câmara de Placas Resistivas (RPC), no âmbito da colaboração R$^3$B no GSI Darmstadt. O trabalho é realizado no contexto da experiência G249, que investiga a estrutura do núcleo rico em neutrões $^{25}$F através de dispersão quase-livre.

%Uma parte central da dissertação é o desenvolvimento de funções de ajuste multidimensional (MDF) no enquadramento do R3BRoot, concebidas para reconstruir grandezas físicas a partir de observáveis dos detetores. Conjuntos de dados simulados foram usados para treinar os modelos MDF, com informação proveniente dos detetores de rastreio FOOT, do dipolo GLAD e da RPC. As funções foram validadas com sucesso em simulações baseadas no Geant4, atingindo elevada precisão preditiva com incertezas da ordem de $10^{-3}$, demonstrando a sua robustez em condições realistas de resolução dos detetores.

%Adicionalmente, foram realizadas simulações dedicadas para estimar a composição de fragmentos que atingem a RPC, identificando os deutérios, trítios e partículas alfa como as contribuições mais relevantes. Estes resultados fornecem informação importante sobre o desempenho esperado do detetor e o seu potencial para o estudo de canais adicionais.

%O desempenho da RPC durante o tempo de feixe da experiência G249 foi avaliado. Resultados principais incluem uma resolução temporal eletrónica de 38 ps, uma eficiência próxima das expectativas e a identificação de tiras problemáticas que afetam a calibração. As aplicações preliminares do método MDF a dados reais mostraram resultados promissores na reconstrução de p/Q, consistentes com as simulações, mas um desempenho mais fraco para A/Q, atribuído à calibração ainda em curso da RPC e dos detetores associados.

%Em suma, este trabalho demonstra o potencial da RPC para evoluir de um detetor puramente temporal e de protões para um dispositivo ativo de reconstrução de fragmentos, ampliando o seu papel na colaboração R$^3$B e abrindo novas perspetivas para a sua implementação em futuras instalações como a FAIR.


Esta dissertação apresenta o desenvolvimento, simulação e aplicação experimental de métodos de reconstrução para a deteção de fragmentos leves utilizando uma Câmara de Placas Resistivas (RPC), no âmbito da colaboração R$^3$B no GSI. O trabalho é realizado no contexto da experiência G249, que investiga a estrutura do núcleo rico em neutrões $^{25}$F através de reações em regime de dispersão quase-livre.

Uma parte central é o desenvolvimento de funções de ajuste multidimensional (MDF) no R3BRoot, concebidas para reconstruir grandezas físicas a partir de observáveis dos detetores. Conjuntos de dados simulados foram usados para treinar os modelos MDF com informação proveniente dos detetores FOOT, do dipolo GLAD e da RPC. As funções foram validadas em simulações baseadas no Geant4, atingindo elevada precisão preditiva com incertezas da ordem de $10^{-3}$, demonstrando robustez em condições realistas.

Simulações adicionais estimaram a composição de fragmentos que atingem a RPC, identificando deutérios, trítios e partículas alfa como dominantes.

O desempenho da RPC durante o tempo de feixe da experiência G249 foi avaliado. Resultados principais incluem uma resolução temporal de 38 ps, eficiência de $\sim$93\% próxima das expectativas e a identificação de tiras problemáticas. As aplicações preliminares do método MDF a dados reais mostraram bons resultados para p/Q mas desempenho inferior para A/Q, atribuído à calibração ainda em curso.

Este trabalho evidencia o potencial da RPC em evoluir de um detetor de protões e tempo para um dispositivo ativo de reconstrução de fragmentos, ampliando o seu papel na R$^3$B e na FAIR.

% Palavras-chave do resumo em Português
% \begin{keywords}
% Palavra-chave 1, Palavra-chave 2, Palavra-chave 3, Palavra-chave 4
% \end{keywords}
\keywords{
  Câmara de Placas Resistivas (RPC) \and
  Dispersão quase-livre (QFS) \and
  Ajuste multidimensional (MDF) \and
  Núcleos exóticos \and
  Colaboração R$^3$B
}
% to add an extra black line
\\
\\
O trabalho de investigação descrito nesta dissertação foi realizado de acordo com as normas
estabelecidas no código de ética da Universidade Nova de Lisboa. O trabalho descrito e o
material apresentado nesta dissertação, com as exceções claramente indicadas, constituem
trabalho original realizado pelo autor.
