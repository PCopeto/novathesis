%!TEX root = ../template.tex
%%%%%%%%%%%%%%%%%%%%%%%%%%%%%%%%%%%%%%%%%%%%%%%%%%%%%%%%%%%%%%%%%%%%
%% acknowledgements.tex
%% NOVA thesis document file
%%
%% Text with acknowledgements
%%%%%%%%%%%%%%%%%%%%%%%%%%%%%%%%%%%%%%%%%%%%%%%%%%%%%%%%%%%%%%%%%%%%

\typeout{NT FILE acknowledgements.tex}%

\begin{ntacknowledgements}

First and foremost, I would like to express my sincere gratitude to my advisor, Professor Daniel Galaviz, not only for his guidance and support throughout this thesis, but also for being my first entry point into the research world of nuclear physics. From the very first time I contacted him three years ago, he welcomed me with open arms into the NUC-RIA group, and for that I am deeply thankful.  

A very special acknowledgment goes to Manuel Xarepe, who in practice was my main mentor during this journey. His patience, availability, and constant willingness to teach made an invaluable difference. I learned immensely from him, and I can only hope that, upon reading this completed thesis, he will finally say to me: “good job”.  

I would also like to warmly thank the entire R$^3$B collaboration for the opportunities to learn and grow, and for immediately making me feel part of the team. I am especially grateful to Valerii Panin, spokesperson of the experiment, for proving that one can be both the smartest and the coolest person in the room at the same time. I would also like to give a special mention to Michael Heil, for his guidance on ToFD; to Prof. Dr. Meytal Duer, for making my Erasmus+ mobility with TU Darmstadt possible; to my friend Beatriz, to my hermanos in Galicia and to Martin Poghosyan, for their support, insights and friendship.  

My thanks also go to my faculty, FCT NOVA, and in particular to the Department of Physics. I gratefully acknowledge the Erasmus+ program, without which my mobility to GSI would not have been possible. I also wish to thank GSI, and in particular the Get\_INvolved program, of which I was fortunate to be a part. Finally, I thank LIP for supporting my participation in the R$^3$B Collaboration Meeting in York, England.  
\\
\\
Now, I also want to thank two great friends that GSI gave me. Hiroshi, for all the deep talks about general relativity, clocks, and Beethoven — somehow managing to make every hang-out a mini-lecture. And Frederik, for sharing not only his friendship but also his friends, and for making me feel like a true Darmstädter.
\\
\\
\\
\\
Finalmente, quero agradecer à minha família e aos meus amigos mais próximos. Aos meus pais, por todo o apoio que me permitiu seguir os estudos que queria, onde queria. Em particular, à minha mãe, que desde cedo me fez estar envolvido com a física. A primeira memória que tenho de me questionar seriamente sobre o universo é, quando ainda muito pequeno, a de a ver deixar cair um sapato e uma folha amarrotada do topo das escadas de casa — e os dois tocarem o chão ao mesmo tempo.

À minha irmã, que em bebé carregava um tijolo a que chamava irmão e que, de certa forma, me fez estar aqui. Aos meus avós, por todas as ajudas “para as pizzas”. E à Guida, a minha madrinha académica não oficial.

Um agradecimento muito especial vai para a minha querida Inês, que sempre esteve ao meu lado nesta jornada, sobretudo nos momentos mais complicados, e que torna sempre tudo mais bonito (só eu é que não consegui tornar Darmstadt mais bonita para ela).

Gostaria ainda de agradecer, do fundo do coração, a TODOS os que fizeram parte do meu percurso académico. 
Em particular, aos meus compaheiros: Grab, Vasco, Dudu e Bia. Ao meu irmoum Minnie. Aos meus padrinhos: Joana e Sparta. Aos meus afilhados: Gabi, Quendera e Bart. E aos Pilares.

Estes últimos anos moldaram-me imenso enquanto pessoa e fizeram nascer em mim um espírito de Estudante ardente, que espero que nunca se extinga.

Tudo tem um fim, ficam as memórias. Novas aventuras virão.

Um grande obrigado e um enorme F.R.A.


\end{ntacknowledgements}