%!TEX root = ../template.tex
%%%%%%%%%%%%%%%%%%%%%%%%%%%%%%%%%%%%%%%%%%%%%%%%%%%%%%%%%%%%%%%%%%%%
%% abstract-en.tex
%% NOVA thesis document file
%%
%% Abstract in English
%%%%%%%%%%%%%%%%%%%%%%%%%%%%%%%%%%%%%%%%%%%%%%%%%%%%%%%%%%%%%%%%%%%%

\typeout{NT FILE abstract-en.tex}%

%This thesis presents the development, simulation, and experimental application of reconstruction methods for light fragment detection using a Resistive Plate Chamber (RPC) in the framework of the R$^3$B collaboration at GSI Darmstadt. The work is carried out in the context of experiment G249, which investigates the structure of the neutron-rich nucleus $^{25}F$ via quasi-free scattering.

%A central part of the thesis is the development of multidimensional fitting (MDF) functions within the R3BRoot framework, designed to reconstruct physical quantities from detector observables. Simulated datasets were used to train the MDF models with input from the FOOT tracking detectors, the GLAD dipole magnet, and the RPC. Functions were successfully validated in Geant4-based simulations, achieving high predictive accuracy with uncertainties down to $10^{-3}$, demonstrating their robustness under realistic detector resolutions.

%In addition, dedicated simulations were performed to estimate the fragment composition reaching the RPC, identifying deuterons, tritons, and alpha particles as the most prominent contributions. These results provide valuable insight into the detector’s expected performance and its potential to study additional physics channels.

%The RPC performance during the G249 beam time is also evaluated. Key results include an electronic time resolution of 38 ps, efficiency close to expectations, and identification of problematic strips affecting calibration. Preliminary applications of the MDF method to real experimental data showed promising results for the reconstruction of p/Q, consistent with simulations, but poorer performance for A/Q, which is attributed to the ongoing calibration of the RPC time-of-flight and associated detectors.

%Overall, this work highlights the RPC’s potential to evolve from a pure timing detector to an active fragment reconstruction device, extending its role within R$^3$B and offering new opportunities for its implementation in future facilities such as FAIR.


This thesis presents the development, simulation, and experimental application of reconstruction methods for light fragment detection using a Resistive Plate Chamber (RPC) in the framework of the R$^3$B collaboration at GSI. The work is carried out in the context of experiment G249, which investigates the structure of the neutron-rich nucleus $^{25}$F via quasi-free scattering.

A central part of the thesis is the development of multidimensional fitting (MDF) functions within the R3BRoot framework, designed to reconstruct physical quantities from detector observables. Simulated datasets were used to train the MDF models with input from the FOOT tracking detectors, the GLAD dipole magnet, and the RPC. Functions were validated in Geant4-based simulations, achieving high predictive accuracy with uncertainties down to $10^{-3}$, demonstrating robustness under realistic detector resolutions.

Simulations were also performed to estimate the fragment composition reaching the RPC, identifying deuterons, tritons, and alpha particles as dominant contributions.

The RPC performance during the G249 beam time was evaluated. Key results include a time resolution of 38 ps, efficiency of $\sim$93\% close to expectations, and identification of problematic strips affecting calibration. Preliminary applications of the MDF method to real data showed promising results for p/Q but poorer resolution for A/Q, attributed to ongoing detector calibration.

Overall, this work highlights the RPC’s potential to evolve from a proton and timing detector to an active fragment reconstruction device, extending its role within R$^3$B and at FAIR.

% Palavras-chave do resumo em Inglês
% \begin{keywords}
% Keyword 1, Keyword 2, Keyword 3, Keyword 4, Keyword 5, Keyword 6, Keyword 7, Keyword 8, Keyword 9
% \end{keywords}
\keywords{
  Resistive Plate Chamber (RPC) \and
  Quasi-free scattering (QFS) \and
  Multidimensional fitting (MDF) \and
  Exotic nuclei \and
  R$^3$B collaboration
}
\\
\\
The research work described in this dissertation was carried out in accordance with the
norms established in the ethics code of Universidade Nova de Lisboa. The work described and
the material presented in this dissertation, with the exceptions clearly indicated, constitute
original work carried out by the author.
