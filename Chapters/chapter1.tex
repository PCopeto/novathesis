%!TEX root = ../template.tex
%%%%%%%%%%%%%%%%%%%%%%%%%%%%%%%%%%%%%%%%%%%%%%%%%%%%%%%%%%%%%%%%%%%
%% chapter1.tex
%% NOVA thesis document file
%%
%% Chapter with introduction
%%%%%%%%%%%%%%%%%%%%%%%%%%%%%%%%%%%%%%%%%%%%%%%%%%%%%%%%%%%%%%%%%%%

\typeout{NT FILE chapter1.tex}%

\chapter{Introduction}
\label{cha:introduction}

\epigraph{
  "The important thing in science is not so much to obtain new facts as to discover new ways of thinking about them."
}{Sir William Lawrence Bragg}

\section{Introduction}

Since the birth of Nuclear Physics, with the discovery of the atomic nucleus by Ernest Rutherford in 1911 \cite{rutherford_lxxix_1911}, this area has proved to be a fascinating field for scientific research displaying a rich variety of quantum phenomena.
Many features of the nucleons in the nucleus exhibited are similar to the structure and behavior of atomic electrons in the atom. Similar descriptions for energy levels and shells, spins and angular momentum have emerged.

But there are some differences:
\begin{enumerate}
	\item The dominating force inside the nucleus is the strong force rather than the electromagnetic one.
	\item Since the strong force is short range and attractive, the potential in which the nucleons exist is created by all the other nucleons in contrast to the force between the atomic electrons and the spatially separated positive charge of the nucleus.
\end{enumerate}

Two fundamental models—the liquid-drop model and the shell model—represent key milestones in this development. Their reconciliation explains many nuclear phenomena, particularly the emergence of magic numbers and the behavior of exotic nuclei.

\subsection{The Liquid-drop Model}

The liquid-drop model, first formulated comprehensively by Weizsäcker and discussed in detail by Bethe and Bacher in 1936 \cite{bethe_nuclear_1936}, treats the nucleus analogously to a charged droplet of incompressible fluid. This model emphasizes collective properties of the nucleus, such as binding energy, surface tension, and Coulomb repulsion among protons.

The semi-empirical mass formula (also called the Bethe-Weizsäcker formula) captures essential trends:

\[B(A,Z) = a_VA - a_SA^{2/3} - a_C\frac{Z(Z-1)}{A^{1/3}} - a_A\frac{(N-Z)^2}{A}\pm \delta(N,Z)\]

where each term accounts for volume, surface, Coulomb, asymmetry, and pairing effects respectively.

While successful at explaining global nuclear properties—such as the approximate binding energy per nucleon—it could not account for observed anomalies in nuclear stability, such as nuclei at specific nucleon numbers (2, 8, 20, 28, 50, 82, 126) exhibiting enhanced stability: the magic numbers.

\subsection{The Shell Model}

The limitations of the liquid-drop model led to the proposal of the shell model, notably formulated by Maria Goeppert Mayer and J. Hans D. Jensen independently in 1949 \cite{mayer_shell_1968}. Their work, expanding on the early suggestions of Elsasser, showed that nucleons move in quantized energy levels within a mean potential well created by all other nucleons—analogous to electrons in atomic orbitals.

Initially, it was thought that a simple three-dimensional harmonic oscillator potential could describe the structure. However, it was soon realized that including a strong spin-orbit coupling term, where the nucleon's spin couples to its orbital motion, was critical to reproduce the magic numbers observed experimentally \cite{haxel_magic_nodate,mayer_shell_1968}.

In particular, spin-orbit splitting lifts the degeneracy of orbital states, energetically favoring high-angular-momentum states (e.g., $j=l+1/2$), thus producing large energy gaps at specific nucleon numbers—those corresponding to the magic numbers.

The modified energy level filling, based on this strong spin-orbit interaction, led to a successful explanation for the pronounced nuclear stability at nucleon numbers:

\[2,8,20,28,50,82,126\]

for both protons and neutrons separately \cite{haxel_magic_nodate,mayer_shell_1968}.


\subsubsection{Magic Numbers and Shell Closures}

In the shell model:

\begin{itemize}
	\item A closed shell means that all available states at a given energy are filled.
	\item Nuclei with both proton and neutron numbers equal to magic numbers (so-called doubly magic nuclei, e.g., $^{16}$O, $^{208}$Pb) exhibit especially high binding energies, spherical shapes, and relatively low excitation spectra.
\end{itemize}

Haxel, Jensen, and Suess \cite{haxel_magic_nodate} provided a succinct explanation showing how a strong spin-orbit coupling splits the energy levels such that filling up the states naturally reproduces the magic numbers.


\subsubsection{Extension to Exotic Nuclei}

The classic shell model was originally built based on stable, near-$\beta$-stable nuclei. However, advances in experimental techniques have allowed the study of exotic nuclei — nuclei far from stability, with unusual neutron-to-proton ratios.

In these systems:

\begin{itemize}
	\item Traditional magic numbers can weaken or even disappear \cite{otsuka_evolution_2020}.
	\item New magic numbers (e.g., $N=16$, $N=34$) can emerge \cite{otsuka_evolution_2020}.
	\item Nuclear deformations become more common, especially near the so-called "island of inversion" (around $N=20$) \cite{otsuka_evolution_2020}.
	\item Phenomena like neutron halos emerge \cite{otsuka_evolution_2020}.
\end{itemize}

This phenomenon has led to the concept of shell evolution, where the shell structure depends on the balance between the nuclear force components (central, spin-orbit, and tensor interactions) and changes with proton-neutron ratios.

\subsection{Conclusion}

In summary, the liquid-drop model offered a macroscopic view of nuclear behavior, while the shell model introduced microscopic structure and quantization effects that explain nuclear stability at magic numbers. The discovery of exotic nuclei has highlighted that the shell structure itself is dynamic and evolves under extreme conditions, demonstrating the richness and complexity of nuclear structure beyond the stable valley.

%\subsection{Liquid-drop Model}
%
%Proposed by Bethe and Bacher in 1936 \cite{bethe_nuclear_1936}.
%
%In this model, nucleons were strongly interacting particles that make up the nucleus, which in turn was considered to be like
%a drop of charged incompressible nuclear liquid.

%However, research showed that certain combinations of protons and neutrons formed more tightly bound nuclei than other \cite{mayer_shell_1968,haxel_magic_nodate}.
%
%\subsection{Shell Model}
%
%\subsubsection{Magic Numbers}
%
%Nuclei can be thought of as having shell structure analogous to the atomic energy levels observed for electrons.
%
%\subsubsection{Exotic Nuclei}
%
%As nuclei became more neutron rich, the nuclear surface becomes more diffuse and the nuclear density decreases.
%\begin{enumerate}
%	\item Nuclear Halos;
%	\item Shell Quenching.
%\end{enumerate}
%
%
%\section{The FAIR Facility}
%
%
%\section{Author's Contribution and Thesis Overview}
